\documentclass[10pt, a4paper]{article}
\usepackage[
a4paper,
total={6.1in, 9.35in},
tmargin=1.0in,
bmargin=0.9in
% total={6.8in, 9.85in},
% bmargin=0.5in
 ]{geometry}
\usepackage{enumitem}% http://ctan.org/pkg/enumitem
\usepackage{outlines}% http://ctan.org/pkg/enumitem
\usepackage{eso-pic} % \AddToShipoutPicture
\usepackage{fontspec}
\usepackage{pgfplots}
\usepackage{hyperref}
% \usepackage[utf8]{inputenc}
% \usepackage{mdframed}
\usepackage{float}
% \usepackage{amsmath}
\usepackage{flafter}
\usepackage{graphicx}
% \usepackage{tikz}
\usepackage{blindtext}
% \usepackage{pgfplots}
\usepackage{seqsplit}
\usepackage[cache=false]{minted}
\usepackage{stmaryrd}
\usepackage{semantic}
\usepackage{url}
\usepackage{color}
\usepackage{flushend}
\usepackage{subfigure}
\usepackage{tikz}
\usepackage{svg}
\usepackage{pdfpages}
\usepackage{titling}
\usepackage{titlesec}
\titlespacing\section{0pt}{14pt plus 4pt minus 2pt}{0pt plus 2pt minus 0pt}
\titlespacing\subsection{0pt}{16pt plus 4pt minus 2pt}{4pt plus 2pt minus 0pt}

\usepackage{cleveref}
\usepackage[parfill]{parskip}
\usepackage[toc,page]{appendix}
\setmainfont{Latin Modern Sans}

\usepackage{tcolorbox}
\usepackage{etoolbox}

\usepackage[strict]{changepage}

\usepackage{fancyhdr}
\newcommand{\sme}{Studies\&Me}
 
\pagestyle{fancy}
\fancyhf{}
% \lhead{Curriculum Vitae for Mikkel Storgaard Knudsen}
\lhead{Curriculum Vitae}
\rhead{Mikkel Storgaard Knudsen}
\begin{document}
\section*{Mikkel Storgaard Knudsen}
\subsection*{Lead Software Engineer}
Bygmestervej 33 5. th. \newline
2400 København NV \newline
\texttt{+}45 42 43 41 45 \newline
mikkelstorgaard@gmail.com
\subsection*{\textit{Summary}}
Experienced software developer with a passion for problem-solving, who finds a great satisfaction in writing modular, testable and \textit{readable} code. I have a strong belief in the mantra "You are only as good as your tools", and a decade of experience writing type-safe, stateless functional programs. \newline
I believe in failing fast, and would rather deal with a hundred annoying compile-time errors than serving any embarrassing runtime error in front of my users. \newline
I am a team player to the core, and as a lead, I find that the training and mentoring of my developers is a core part of job. Likewise, I myself thoroughly enjoy working together with more experienced people who I can learn from as well. I especially thrive with cross-team work -- something I benefit from greatly in my current position at \sme, where I work together with medical specialists, UX designers and data scientists to make clinical studies accessible for interested participants all over the globe.

\subsection*{\textit{Experience}}
\subsection*{Lead Engineer of Decentralized Clinicals Trials at \sme}
\subsubsection*{{\normalfont(September 2020 -- present)}}
\textbf{Technical responsibilities- and achivements}
\begin{outline}
  \1 The design and implementation of \sme's own cross-platform DCT whitelabel app.\newline
    {\footnotesize \textbf{Dart, Flutter, Drift/SQLite3}, plus \textbf{Kotlin} and \textbf{Swift} for native modules and external integrations}

  \1 Design/implementation of the backend server for the above-mentioned app.\newline
  {\footnotesize \textbf{Scala 2, Amazon Web Services (Lambda, Elastic Beanstalk, Elastic Container Registry, S3, Secrets Manager)}}

  \1 Directing/implementing the tests for both the app and the backend.\newline
    {\footnotesize \textbf{Integration- and unit tests (including mock- and property based testing)}}

  \1 Primary responsible for continuous integration- and deployment of both app and server.\newline
    {\footnotesize \textbf{CircleCI/GitHub}, automated building- and App Store deployment using \textbf{Fastlane}}

  \1 Screening, interviewing and hiring new developers for \sme, as well as leading and managing a team of six developers.
\end{outline}

During our collaboration with SnapIoT (April 2020 -- September 2021), I furthermore
\begin{outline}
  \1 Reverse engineered SnapIoT's closed-source SDK, providing our developers with missing tools (such as a debug console), and adding missing (but trial-critical) features to the SnapIoT's existing framework.

  \1 Designing and implemented a template-based framework in SnapIoT's existing toolset, separating trial content (i.e. questionnaires, onboarding flows) from app code, greatly reducing the amount of hardcoded app screens and enabling code reuse.
  
  \1 Trained SnapIoT's own in-house developers in using the framework mentioned above.
\end{outline}
\vspace*{\fill}
\textit{(Cont'd)}

\pagebreak

\subsubsection*{Stakeholder responsibilites}
As Lead Engineer, I have regularly represented \sme\ in stakeholder meetings across almost all aspects of our DCT developments, in meetings such as 
\begin{outline}
  \1 Specification/planning meetings with project owners- and data science leads from study sponsors such as LEO Pharma and Novo Nordisk.

  \1 Specification/planning meetings with various vendors and suppliers such as Klifo, 4G Clinical and Withings.

  \1 Virtual- and on-site meetings with close collaborator SMART-TRIAL, for planning and designing new features for their main Electronic Data Capture platform.
\end{outline}
\subsubsection*{In-house responsibilites}
\begin{outline}
  \1 Member of \sme's Environment, Health \& Safety group, partaking in workplace assessments (\textit{Arbejdspladsvurdering}) evaluations, and designing and implementing action plans for alleviating identified issues. I have also taken and completed the mandatory \textit{Arbejdsmiljøuddannelse}.

\1 Together with our compliance department, I have been responsible for
  \2 Designing our \textit{GAMP5}-based Quality Management System (QMS), ensuring that our software development is auditable, and that products are validated and proven to follow the international \textit{Good Clinical Practice} guidelines. The QMS been audited by both Klifo and MedicQA.
  \2 Implementing above-mentioned QMS in tools including \textbf{Confluence, TestRail and Jira}.

\1 Responsible for administrating business-critical services such as Google Workspace (Mail, Drive, Groups, etc.), Slack and 1Password.
\end{outline}

\subsubsection*{\textit{Experience, continued}}
\textbf{Software developer at Studies\&Me} \textit{(March 2020 -- August 2020)}
\begin{outline}
\1 Responsible for continued Studies\&Me backend development, -design and maintenance, including tasks in database design and development operations\newline
    {\footnotesize \textbf{Scala 2, MySQL, GraphQL, CircleCI, various AWS services}}
\1 Quickly transitioned to focus full time on  Decentralized Clinical Trials 
\end{outline}

\textbf{Software developer in Business Intelligence at Abacus Medicine} \textit{(August 2019 -- March 2020)}
  \begin{outline}
  \1 Main feature developer on in-house \textit{assisted decision-making} trading platform.\newline
  {\footnotesize \textbf{OCaml, C\#, T-SQL, Docker, Azure, Jira, Bitbucket}}

  \1 Tasks included include front- and backend design, testing and
    implementation, and tasks in our continuous integration and -deployment pipeline.

  \1 Weekly meetings with internal stakeholders and developers, collaborating
    on resolving design problems, prioritizing new features, and roadmapping. 
  \end{outline}

\textbf{Backend developer at June by Danske Bank} \textit{(March 2017 -- February 2019)}
\begin{outline}
  \1 Primary responsible for designing and implementing a new image recognition engine for verifying customer identity documents during onboarding, in alignment with Danske Bank's \textit{Know Your Customer} principles. I raised the accuracy from 56\% to \textasciitilde 92\%, compared to their existing solution.\newline
    {\footnotesize \textbf{Python2, OpenCV, Tesseract OCR, Docker}}

  \1 Part of the feature development team in June, with responsibilities in software implementation, code reviews, sprint planning and ticket grooming sessions.\newline
    {\footnotesize \textbf{C\#, Docker, Microsoft SQL Server}}
\end{outline}

\textbf{Full-stack Junior Consultant at Eksponent} \textit{(March 2016 -- March 2017)}
\begin{outline}
\1 Design-, development- and deployment responsibilities for client projects, as well as tool development for internal usage.\newline
  {\footnotesize \textbf{C\#, JavaScript, Microsoft SQL Server}}
\1 Design- and planning meetings with clients such as Styrelsen for Patientsikkerhed, Københavns Kommune, and others.
\end{outline}

\textbf{Teaching Assistant at The University of Copenhagen} \textit{(September 2017 -- November 2017)}
\begin{outline}
\1 Teaching assistant on the \textit{Advanced Programming} master's course. Tasks included preparing and hosting weekly lab sessions, correcting and grading assignments, and assisting in grading the course's final exam submissions.\newline
  {\footnotesize\textbf{Haskell, Prolog, Erlang}}
\end{outline}

\subsection*{\textit{Education}}
\textbf{MSc in Computer Science at The University of Copenhagen} \textit{(2016 -- 2019)}
\begin{outline}
  \1 Study programme focused on compiler design, semantics and types and formal logic, and on parallel programming, including GPGPU programming in CUDA/C++.

  \1 Projects included assisting a Ph.D. student in Physics in rewriting a complex three-dimensional microbiological simulation from a sequentially executed model to a parallel GPU-powered ditto, vastly increasing computational times while decreasing code complexity.
  
  \1 For my thesis \textit{FShark: Futhark programming in FSharp}, my two biggest contributions was the design and implementation of a C\# backend for the Futhark compiler, and the design and implementation of an F\#-to-Futhark transpiler. The C\# backend became part of the official Futhark compiler, but was since deprecated as there was little demand for maintaining the .NET-libraries.

  \1 My project was graded 12 (\textit{A on the intl. ECTS scale}) at the thesis defense.
\end{outline}
  
\textbf{BSc in Computer Science at The University of Copenhagen} \textit{(2012 -- 2016)}
\begin{outline}
\1 Bachelor's thesis: Extending the Futhark programming language with working type aliases, as well as describing an implementation of ML-style higher-order modules for Futhark.
\end{outline}

\textbf{General Certificate of Secondary Education (STX) at Viborg Gymnasium \& HF} \textit{(class of 2010)}
\begin{outline}
\1 English, mathematics and social sciences
\end{outline}

\subsection*{\textit{Other experience}}
\begin{outline}
  \1 Board member in the student-driven \textit{Kantinen} at UCPH (2013 -- 2018)\newline
    {\footnotesize Tasks included for daily maintenance, event planning- and execution, keeping inventory and placing weekly orders with B2B wholesalers like Sjøbeck, Carlsberg and Sugro.}

  \1 Member of \textit{DIKUrevy} at UCPH (2013 -- 2019)\newline
    {\footnotesize Writing/directing sketches and songs, on-stage performances in front of large (400\texttt{+}) audiences, event- and party-planning, plus two years as co-chair of the organization.}

  \1 Board member in \textit{BLUS} (Copenhagen's LGBT+ Student Association)  (2012 -- 2015)\newline
    {\footnotesize Tasks included planning and hosting parties and weekly events at \textit{Huset i Magstræde}, and collaborations with peer organizations.}
\end{outline}
\vspace*{\fill}
\subsection*{\textit{About me}}
On a more personal level, I see myself as an outgoing and open-minded individual.
I care a lot about the environment around me and others, and I involve myself in my workplace's culture. I like social events, and consistently volunteer myself in planning committies whether it's for the occational Friday bar, or the large annual company-wide parties.
I rarely decline a challenge -- especially not if it's within my interests. Case in point, I applied for Mensa's admission test in 2021 on a dare, and passed with flying colors.

I have an undying love for music, and love the Copenhagen nightlife. I have DJ'ed for several years, and love clubbing, going to concerts, to the cinema, or just about any other kind of show.
In my remaining spare time, I play a bit of bass. I play badminton twice a week, and enjoy cooking on a hobby level.

\vspace*{\fill}
\begin{flushright}
  Last updated \today
\end{flushright}
\end{document}


%%% Local Variables:
%%% coding: utf-8
%%% TeX-command-extra-options: "-shell-escape"
%%% mode: latex
%%% TeX-engine: xetex
%%% End:
