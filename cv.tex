\documentclass[10pt, a4paper]{article}
\usepackage[
a4paper,
total={6.1in, 9.35in},
tmargin=1.0in,
bmargin=0.9in
% total={6.8in, 9.85in},
% bmargin=0.5in
 ]{geometry}
\usepackage{enumitem}% http://ctan.org/pkg/enumitem
\usepackage{outlines}% http://ctan.org/pkg/enumitem
\usepackage{eso-pic} % \AddToShipoutPicture
\usepackage{fontspec}
\usepackage{pgfplots}
\usepackage{hyperref}
% \usepackage[utf8]{inputenc}
% \usepackage{mdframed}
\usepackage{float}
% \usepackage{amsmath}
\usepackage{flafter}
\usepackage{graphicx}
% \usepackage{tikz}
\usepackage{blindtext}
% \usepackage{pgfplots}
\usepackage{seqsplit}
\usepackage[cache=false]{minted}
\usepackage{stmaryrd}
\usepackage{semantic}
\usepackage{url}
\usepackage{color}
\usepackage{flushend}
\usepackage{subfigure}
\usepackage{tikz}
\usepackage{svg}
\usepackage{pdfpages}
\usepackage{titling}
\usepackage{titlesec}
\titlespacing\section{0pt}{14pt plus 4pt minus 2pt}{0pt plus 2pt minus 0pt}
\titlespacing\subsection{0pt}{16pt plus 4pt minus 2pt}{4pt plus 2pt minus 0pt}

\usepackage{cleveref}
\usepackage[parfill]{parskip}
\usepackage[toc,page]{appendix}
\setmainfont{Latin Modern Sans}

\usepackage{tcolorbox}
\usepackage{etoolbox}

\usepackage[strict]{changepage}

\usepackage{fancyhdr}
\newcommand{\sme}{Studies\&Me}
 
\pagestyle{fancy}
\fancyhf{}
% \lhead{Curriculum Vitae for Mikkel Storgaard Knudsen}
\lhead{Curriculum Vitae}
\rhead{Mikkel Storgaard Knudsen}
\begin{document}
\section*{Mikkel Storgaard Knudsen}
\subsection*{Software Developer}
Bygmestervej 33 5. th. \newline
2400 København NV \newline
\texttt{+}45 42 43 41 45 \newline
mikkelstorgaard@gmail.com
\subsection*{\textit{Summary}}
Experienced software developer with a passion for problem-solving. I'm a strong believer in "the right tool for the job", and have years of experience writing modular, testable and readable code in both functional and imperative languages.
Code safety is important to me, and I don't believe a code base is complete unless it's accompanied by a thorough test stack.

I'm a team player, and I enjoy training new colleages, just as I enjoy working with- and learning from more experienced colleagues as well.
I like working across teams, and am experienced in designing good solutions in collaboration with UX-, frontend-, localization teams et al.

I thrive in international organizations, and I thoroughly enjoy working together with people from across the globe.

\subsection*{\textit{Experience}}
\subsection*{Software Engineer II at Zendesk (Explore)}
\subsubsection*{{\normalfont(September 2022 -- May 2025)}}
At Zendesk, I was part of team \textit{Perseverance} -- the team responsible for Explore's backend query compilation- and execution engine. 
For context: Explore is Zendesk's analytics application. It allows data owners to generate complete reports over their entire data collection, through simple drag-and-drop interfaces.

Besides the usual day-to-day operations and feature additions, my role in Perseverance regularly included 
\begin{outline}
 \1 Complex bug investigations, often spanning several days, spread across multiple services and production environments
 \1 Frequent collaboration on shared projects with other dev teams within the org.
 \1 Training new members of the team.
 \1 Technical presentations on Explore's internal town hall meetings
\end{outline}
Furthermore, I've been a part of Explore's on-call rotation since January 2024

\textbf{Three contributions I'm particularly proud of during my time at Zendesk:}
\begin{outline}
\1 Co-authored  advanced type checking in Explore's custom formula language BIME
\1 Authored advanced syntax checking/error reporting for BIME
\1 Sole responsible for designing and implementing multiple mission critical integrations between Explore and Zendesk's main infrastructure
\end{outline}
    {\footnotesize At Zendesk, my main toolset has been\newline \textbf{Scala, Ruby, Amazon (RDS, Redshift, ECS and ECR, CloudWatch, S3, EKS and MSK), BigQuery and DataDog}}

\subsection*{Lead Engineer of Decentralized Clinicals Trials at \sme}
\subsubsection*{{\normalfont(September 2020 -- June 2022)}}
\textbf{Technical responsibilities- and achivements}
\begin{outline}
  \1 The design and implementation of \sme's own cross-platform DCT whitelabel app.\newline
    {\footnotesize \textbf{Dart, Flutter, Drift/SQLite3}, plus \textbf{Kotlin} and \textbf{Swift} for native modules and external integrations}

  \1 Design/implementation of the backend server for the above-mentioned app.\newline
  {\footnotesize \textbf{Scala 2, Amazon Web Services (Lambda, Elastic Beanstalk, Elastic Container Registry, S3, Secrets Manager)}}

  \1 Directing/implementing the tests for both the app and the backend.\newline
    {\footnotesize \textbf{Integration- and unit tests (including mock- and property based testing)}}

  \1 Primary responsible for continuous integration- and deployment of both app and server.\newline
    {\footnotesize \textbf{CircleCI/GitHub}, automated building- and App Store deployment using \textbf{Fastlane}}

  \1 Screening, interviewing and hiring new developers for \sme, as well as leading and managing a team of six developers.
\end{outline}

\textbf{Three highlights from our collaboration with SnapIoT (April 2020 -- September 2021)}:
\begin{outline}
  \1 Reverse engineered SnapIoT's closed-source SDK, providing our developers with missing tools (such as a debug console), and adding missing (but trial-critical) features to the SnapIoT's existing framework.
  \1 Designing and implemented a template-based framework in SnapIoT's existing toolset, separating trial content (i.e. questionnaires, onboarding flows) from app code, greatly reducing the amount of hardcoded app screens and enabling code reuse.
  \1 Trained SnapIoT's own in-house developers in using the framework mentioned above.
\end{outline}

\subsubsection*{Stakeholder responsibilites}
As Lead Engineer, I have regularly represented \sme\ in stakeholder meetings such as 
\begin{outline}
  \1 Specification/planning meetings with project owners- and data science leads from study sponsors such as LEO Pharma and Novo Nordisk.
  \1 Specification/planning meetings with various vendors and suppliers such as Klifo, 4G Clinical and Withings.
  \1 Virtual- and on-site meetings with close collaborator SMART-TRIAL, for planning and designing new features for their main Electronic Data Capture platform.
\end{outline}

\subsubsection*{In-house responsibilites}
\begin{outline}
  \1 Member of \sme's Environment, Health \& Safety group, partaking in workplace assessments (\textit{Arbejdspladsvurdering}) evaluations, and designing and implementing action plans for alleviating identified issues. I have completed the mandatory \textit{Arbejdsmiljøuddannelse}.

\1 Together with our compliance department, I have been responsible for
  \2 Designing our \textit{GAMP5}-based Quality Management System (QMS), ensuring that our software was proven to follow the international \textit{Good Clinical Practice} guidelines. The QMS been audited by both Klifo and MedicQA.
  \2 Implementing above-mentioned QMS in tools including \textbf{Confluence, TestRail and Jira}.

\1 Responsible for administrating business-critical services such as Google Workspace (Mail, Drive, Groups, etc.), Slack and 1Password.
\end{outline}
\pagebreak

\subsubsection*{\textit{Experience cont'd}}
\begin{outline}

\1 \textbf{Software developer at Studies\&Me} \textit{(March 2020 -- August 2020)}
\2 Responsible for continued Studies\&Me backend development, -design and maintenance, including tasks in database design and development operations\newline
    {\footnotesize \textbf{Scala 2, MySQL, GraphQL, CircleCI, various AWS services}}
\2 Quickly transitioned to focus full time on Decentralized Clinical Trials 

\1 \textbf{Software developer in Business Intelligence at Abacus Medicine} \textit{(August 2019 -- March 2020)}
  \2 Main feature developer on in-house \textit{assisted decision-making} trading platform.\newline
  {\footnotesize \textbf{OCaml, C\#, T-SQL, Docker, Azure, Jira, Bitbucket}}

\1 \textbf{Backend developer at June by Danske Bank} \textit{(March 2017 -- February 2019)}
  \2 Primary responsible for designing and implementing a new image recognition engine for identity document verification, raising accuracy from 56\% to \textasciitilde 92\% compared to the existing solution.\newline
    {\footnotesize \textbf{Python2, OpenCV, Tesseract OCR, Docker}}

  \2 Regular feature development and backend work.\newline
    {\footnotesize \textbf{C\#, Docker, Microsoft SQL Server}}

\1 \textbf{Full-stack Junior Consultant at Eksponent} \textit{(March 2016 -- March 2017)}
\2 Design-, development- and deployment responsibilities for client projects, as well as tool development for internal usage.\newline
  {\footnotesize \textbf{C\#, JavaScript, Microsoft SQL Server}}
\2 Stakeholder meetings with clients such as Styrelsen for Patientsikkerhed, Københavns Kommune, et al.

\1 \textbf{Teaching Assistant at The University of Copenhagen} \textit{(September 2017 -- November 2017)}
\2 Teaching assistant on the \textit{Advanced Programming} master's course. Tasks included preparing/hosting lab sessions, correcting and grading assignments, and grading assistance on the final exam submissions.\newline
  {\footnotesize\textbf{Haskell, Prolog, Erlang}}
\end{outline}

\subsection*{\textit{Education}}
\textbf{MSc in Computer Science at The University of Copenhagen} \textit{(2016 -- 2019)}
\begin{outline}
  \1 Study programme focused on compiler design, semantics and types and formal logic, and on parallel programming, including GPGPU programming in CUDA/C++.

  \1 Projects included assisting a Ph.D. student in Physics in rewriting a complex three-dimensional microbiological simulation from a sequentially executed model to a parallel GPU-powered ditto, vastly increasing computational times while decreasing code complexity.
  
  \1 For my thesis \textit{FShark: Futhark programming in FSharp}, my two biggest contributions was the design and implementation of a C\# backend for the Futhark compiler, and the design and implementation of an F\#-to-Futhark transpiler. The C\# backend became part of the official Futhark compiler, but was since deprecated as there was little demand for maintaining the .NET-libraries.

  \1 My project was graded 12 (\textit{A on the intl. ECTS scale}) at the thesis defense.
\end{outline}
  
\textbf{BSc in Computer Science at The University of Copenhagen} \textit{(2012 -- 2016)}
\begin{outline}
\1 Bachelor's thesis: Extending the Futhark programming language with working type aliases, as well as describing an implementation of ML-style higher-order modules for Futhark.
\end{outline}

\textbf{General Certificate of Secondary Education (STX) at Viborg Gymnasium \& HF} \textit{(class of 2010)}
\begin{outline}
\1 English, mathematics and social sciences
\end{outline}

\subsection*{\textit{Other experience}}
\begin{outline}
  \1 Board member in the student-driven \textit{Kantinen} at UCPH (2013 -- 2018)\newline
    {\footnotesize Tasks included for daily maintenance, event planning- and hosting, inventory management and weekly purchasing from B2B wholesalers.}

  \1 Member of \textit{DIKUrevy} at UCPH (2013 -- 2019)\newline
    {\footnotesize Writing/directing sketches and songs, on-stage performing for large (400\texttt{+}) audiences, event- and party-planning, plus two years as co-chair of the organization.}

  \1 Board member in \textit{BLUS} (Copenhagen's LGBT+ Student Association)  (2012 -- 2015)\newline
    {\footnotesize Tasks included planning and hosting parties and weekly events at \textit{Huset i Magstræde}, and collaborations with peer organizations.}
\end{outline}
\vspace*{\fill}
\subsection*{\textit{About me}}
On a personal level, I see myself as a social and open-minded individual.
I like social events, and consistently volunteer myself in planning committies whether it's for the occational Friday bar, or the large annual company-wide parties.
I love music and love the Copenhagen nightlife in general, whether it's clubbing or going to concerts.
I play badminton three nights per week, and have played both Danish and international tournaments.
\vspace*{\fill}
\begin{flushright}
  Last updated \today
\end{flushright}
\end{document}


%%% Local Variables:
%%% coding: utf-8
%%% TeX-command-extra-options: "-shell-escape"
%%% mode: latex
%%% TeX-engine: xetex
%%% End:
